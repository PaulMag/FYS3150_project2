A real symmetrix matrix can through repeated similarity
transformations be written as a strictly diagonal matrix.
Similarity transformations are accomplished by using a sparse
transformation matrix $S$, where the non-zero elements are
\begin{align*}
    s_{kk} &= s_{ll} = \cos{(\theta)}, \\
    s_{kl} &= -s_{lk} = -\sin{(\theta)}, \\
    s_{ii} &= 1 \text{, where } i = 1,2,3,\dots,n,
\end{align*}
which in $n$-dimensional Euclidean space, performs a plane rotation
around an angle $\theta$. One similarity transformation of a matrix
$A$ is then calculated by multiplying with $S$ and its transpose,
\begin{align*}
    B = S^T A S.
\end{align*}
The elements of the matrix $B$ will after the multiplication be
decided by the following algorithm.
\begin{algorithmic}
    \For{i=1,2,3,\dots,n}
        \State$b_{ii} = a_{ii}$

        \State $b_{ik} = a_{ik}\cos{(\theta)} - a_{il}\sin{(\theta)}$

        \State $b_{il} = a_{il}\cos{(\theta)} + a_{ik}\sin{(\theta)}$

        \State $b_{kk} = a_{kk}\cos^2{(\theta)} -
        2a_{kl}\cos{(\theta)}\sin{(\theta)} +
        a_{ll}\sin^2{(\theta)}$

        \State $b_{ll} = a_{ll}\cos^2{(\theta)} -
        2a_{kl}\cos{(\theta)}\sin{(\theta)} +
        a_{kk}\sin^2{(\theta)}$
        
        \State $b_{kl} =
        (a_{kk}-a_{ll})\cos{(\theta)}\sin{(\theta)} +
        a_{kl}(\cos^2{(\theta)} - \sin^2{(\theta)})$
    \EndFor
\end{algorithmic}
To employ this to work in our favour --- to actually gradually
reduce nondiagonal elements, the choice of $\theta$ must be a
specific number. The number can be found by requiring that \[b_{kl}
= b_{lk} = 0.\] This gives us the equation \[b_{kl} = (a_{kk} -
a_{ll})\cos{(\theta)}\sin{(\theta)} + a_{kl}(\cos^2{(\theta)} -
\sin^2{(\theta)}) = 0,\] which when introducing the variable \[\tau
= \frac{a_{ll}-a_{kk}}{2a_{kl}},\] can be rewritten as \[t^2 +
2\tau t - 1 = 0.\] where $t = \tan{(\theta)}$. This means that
$t$ can be found by \[t = -\tau \pm \sqrt{1 + \tau^2} =
\frac{1}{\tau \pm \sqrt{1 + \tau^2}}.\] The smallest of the two
roots should be chosen from this expression.

<+Explenation of how theta is kept below threshold and why.+>

Furthermore, the values \[c = \cos{(\theta)} = \frac{1}{\sqrt{1 +
t^2}}\] and $s = \sin{(\theta)} = tc$ can be calculated.
