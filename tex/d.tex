% paulms:

% Task d: Plot the wave function for two electrons as functions of the relative
% coordinate r.

We want to plot the wave functions, or more specifically, the
probability functions for two electrons inside the harmonic
oscillator. To do this one needs to find the eigenvectors that
correspond to each eigenvalue of the system.

The Jacobi Algorithm we used correspond to doing
$$ B = S^T_N \cdots S^T_3 \\ S^T_2 \\ S^T_1 \\ A \\ S_1 \\ S_2 \\
S_3 \cdots S_N = S^T A S, $$
where $N$ is the number of iterations that was required to make all
the non-diagonal elements of $B$ be $\approx 0$ and $S$ is the
eigenvectors, or eigenfunctions, that correspond to the
eigenvectors in $B$. To find just the $S$ we simply need to perform
a part of the Jacobi rotation on the unit matrix $I_n$. After each
multiplication the elements of $S$ will be changed according to
this algorithm:
\begin{algorithmic}
    \For{i = 1,2,3,\dots,n}
        \State$r_{i,k} = cos(\theta) * r_{i,k} - sin(\theta) * r_{i,l}$
		\State$r_{i,l} = cos(\theta) * r_{i,l} + sin(\theta) * r_{i,k}$
    \EndFor
\end{algorithmic}
The columns in the resulting matrix $S$ is then the eigenvectors
that correspond to the eigenvalues in the diagonal of $B$, so we
will receive $n$ eigenvectors of length $n$. We are only interested
in finding the wave functions for the ground state and the two
lowest excited states: $\lambda_0$, $\lambda_1$ and $\lambda_2$. To
find the eigenvectors that correspond to these eigenvalues we need
to find which columns in $B$ that contains these eigenvalues and
pick out the columns in $S$ with the same position.

We will visualize these eigenvectors by plotting the actual
probability distributions. A Python script is used to read and
visualize these vectors. If we call the vectors $\mathbf{v}_i$ the
actual wave functions are $\psi_i(\rho) =
\frac{\mathbf{v}_i}{\rho}$ and the probability distributions is
$P(\rho) = \psi_i(\rho)^2 = ( \frac{\mathbf{v}_i}{\rho} )^2$

Lastly the vectors are normalized so that the total probability
will appear as one for each energy state.
$$
\mathrm{normalized}(\mathbf{v}_i)
= \frac{\mathbf{v}_i}{\mathrm{d}\rho \cdot \sum \mathbf{v}_i }
= \frac{n \cdot \mathbf{v}_i}{\rho \cdot \sum \mathbf{v}_i }
$$
