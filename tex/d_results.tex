% paulms:

\bildeto{images/prob_1}{images/prob_2}{fig:prob12}{Probability
distribution for the three lowest energy states for two electrons
in two different weak potentials.}
\bildeto{images/prob_3}{images/prob_4}{fig:prob34}{Two different
stronger potentials.}

From figures \ref{fig:prob12} and \ref{fig:prob34} we see how the
electrons behave in oscillators of different potential strengths.
Note that the radius $\rho$ is different on each figure. For each
different $\omega_r$ $\rho_{max}$ was selected so that the wave
function settled at $\approx 0$ \emph{before} $\rho_{max}$. The
ground state ($E_0$) has one top, the first excited state ($E_1$)
has two tops, $E_2$ has three top, etc.

For higher $\omega_r$ the ``tops'' are not necessarily extremum
points, but rather ``humps'' in the slopes, and it it is only when
$\omega_r$ that we can see a decrease in probablity near $\rho = 0$
where the electrons are close together. One could expect a more
distinct valley here, but it seems the stronger potentials dominate
the Coulomb force.

The strange behavior at the far left side is probably just due to
the sloppy method used to avoid division by zero.

\bildeto{images/prob_0_short}{images/prob_0_long}{fig:prob0}{Two
    electrons in oscillators of different sizes with \emph{no}
potential strength at all.}

In figure \ref{fig:prob0} there are only two ``free'' electrons.
Obviously from the figures, they still appear to be in a potential,
but no matter how large we set $\rho_{max}$ they will use all the
space available, since the only force present is the Coulomb force
that is pushing them away from eachother. This strange behavior
occurs because our model will always set the probability to 0 at
$\rho_{max}$ and not consider the possibility that there is some
probability of existance beyond this.
