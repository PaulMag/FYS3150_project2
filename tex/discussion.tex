Implementing the Jacobi rotation algorithm have given us great insight in how some
eigenvalue solvers work. The Jacobi method is easy to implement and understand. The
drawback is that it repeatedly needs to transform elements that were previously set to
zero. Therefore it grows slow for large matrices. For future projects, built-in functions
such as Armadillo's \texttt{eig\_sym()} are much easier and faster to use, at no expense to
the precision.

The eigenvalue method for solving this problem has some limitations. If $\rho_{max}$ is too small, the shape of the wavefunctions will compress to fit into the given interval and appear to be correct, even though the width should be much larger. This led to some confusion, but of course, when one is aware of this choosing the correct $\rho_{max}$ is not that difficult. When the Coulomb force was introduced and we experimented with removing the potential completely the wavefunctions we got still showed the behavior of being in a potential, because of the problem with $\rho_{max}$. Of course, the situations where $\omega_r = 0$ is trivial and not very interesting, so we do not need this algorithm it solve it.
