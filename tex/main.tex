\documentclass[a4paper,english]{article}

\usepackage[utf8]{inputenc}
\usepackage[english]{babel}
\usepackage{graphicx,color}
\usepackage{amsmath}
\usepackage[small,bf]{caption}
\usepackage{hyperref}
\usepackage{fancyhdr}
\usepackage{algpseudocode}
\usepackage{parskip}
\usepackage{caption}

% Only for one purpose: Placeholder text
\usepackage{lipsum}

% Packages for including code
\usepackage{listings}
\lstset{language=idl} % or python, java, etc\ldots
\lstset{basicstyle=\ttfamily\scriptsize} % \small if short code
\lstset{frame=single} % creates the frame around code
\lstset{title=\lstname} % display name of file, not necessary
\lstset{keywordstyle=\color{red}\bfseries}
\lstset{commentstyle=\color{blue}}
\lstset{stringstyle =\color{darkgreen}}
\lstset{showspaces=false}
\lstset{showstringspaces=false}
\lstset{showtabs=false}
\lstset{breaklines=true}
\lstset{tabsize=4}

% Commands for referencing
\newcommand{\refeq}[1]{(\textcolor{red}{\ref{eq:#1}})} % Red color when referencing equations.
\newcommand{\refig}[1]{\textcolor{blue}{\ref{fig:#1}}} % Blue color when referencing figures.
\newcommand{\reflst}[1]{(\textcolor{red}{\ref{lst:#1}})}
\newcommand{\reftab}[1]{\textcolor{blue}{\ref{tab:#1}}} % Blue color when referencing tables.

% Fancy ruler needed for titlepage
\newcommand{\HRule}{\rule{\linewidth}{0.5mm}}

\newcommand{\bilde}[3]{
	\begin{figure}[htbp]
		\centering
		\includegraphics[width=\textwidth]{#1}
		\caption{#3 \label{#2}}
	\end{figure}
}
\newcommand{\bildeto}[4]{
	\begin{figure}[htbp]
		\centering
		\includegraphics[width=0.96\textwidth]{#1}
		\includegraphics[width=0.96\textwidth]{#2}
		\caption{#4 \label{#3}}
	\end{figure}
}

\author{Kristoffer Brækken, Vedad Hodzic, Paul Magnus
Sørensen-Clark}

\begin{document}

\begin{titlepage}
    \thispagestyle{empty}
    % Explanation of \\[]-syntax:
% In the square brackets you can define a certain vertical space
% which is inserted in the text. Convenient for fancy formatting.

\begin{center}
    \LARGE University of Oslo\\[1.5cm]
    \Large FYS3150 - Project 2 \\ Computational physics\\[0.5cm]

    \HRule \\[0.4cm]

    % Title of project
    { \huge \bfseries Solving Schrödinger's equation in 3D as
    eigenvalue problem\\[0.4cm] }

    \HRule \\[1.5cm]

    % Spaces between lines are there for line breaks
    \large Kristoffer Brækken, \emph{jaremikb}

    \large Paul Magnus Sørensen-Clark, \emph{paulms}

    \large Vedad Hodzic, \emph{vedadh}

    \vfill

    {\large \today}
\end{center}

\end{titlepage}

\section{Introduction}
Schrödinger's equation is used to describe the nature of particles
on the quantum scale. In this project we will look at Schrödinger's
equation for both one and two electrons in a three-dimensional
harmonic oscillator potential with and without the coulomb
interaction.

We will rewrite Schrödinger's equation in terms of dimensionless
variables and rearrange it into an eigenvalue problem. The
eigenvalues of a real symmetric matrix can be found by reducing the
non-diagonal matrix elements via continuously applying
similarity-transformation until they are all below a certain
predetermined tolerance $\epsilon$.

Furthermore, we will perform an analysis on the exactness of this
method with respect to computational cost.


\section{Theory}
Our intent is to find the solution of the radial part of the
Schrödinger equation, as seen in equation \ref{eq:schrod}.  This
can be done first for one electron and then extended to describe
two electrons including their Coulomb interaction.

\begin{equation}
    -\frac{\hbar^2}{2m}\left(
    \frac{1}{r^2}\frac{d}{dr}r^2\frac{d}{dr} -
    \frac{l(l+1)}{r^2} \right)R(r) + V(r)R(r) = ER(r)
    \label{eq:schrod}
\end{equation}

Via introducing dimensionless variables for the radius \[\rho =
\frac{1}{\alpha}r,\] and \[R(r) = \frac{1}{r}u(r),\] the equation
can be rewritten as

\begin{equation}
    -\frac{d^2}{d\rho^2}u(\rho) + \rho^2u(\rho) = \lambda
    u(\rho).
    \label{eq:dimensionless_schrod}
\end{equation}

Which converts the problem of solving the radial Schrödinger
equation into an eigenvalue problem where \[\lambda =
\frac{2m\alpha^2}{\hbar^2}E,\] are the eigenvalues we are
interested in.


\section{Algorithm}
Our intent is to find the solution of the radial part of the
Schrödinger equation, as seen in equation \ref{eq:schrod}.  This
can be done first for one electron and then extended to describe
two electrons including their Coulomb interaction.

\begin{equation}
    -\frac{\hbar^2}{2m}\left(
    \frac{1}{r^2}\frac{d}{dr}r^2\frac{d}{dr} -
    \frac{l(l+1)}{r^2} \right)R(r) + V(r)R(r) = ER(r)
    \label{eq:schrod}
\end{equation}

Via introducing dimensionless variables for the radius \[\rho =
\frac{1}{\alpha}r,\] and \[R(r) = \frac{1}{r}u(r),\] the equation
can be rewritten as

\begin{equation}
    -\frac{d^2}{d\rho^2}u(\rho) + \rho^2u(\rho) = \lambda
    u(\rho).
    \label{eq:dimensionless_schrod}
\end{equation}

Which converts the problem of solving the radial Schrödinger
equation into an eigenvalue problem where \[\lambda =
\frac{2m\alpha^2}{\hbar^2}E,\] are the eigenvalues we are
interested in.


\section{Plot the wave function for two electrons in oscillator}
% paulms:

% Task d: Plot the wave function for two electrons as functions of the relative
% coordinate r.

I want to plot the wave functions, or more specifically, the probability
functions for two electrons inside the harmonic oscillator.
To do this I need to find the eigenvectors that correspond to each eigenvalue
of the system. I will only find the wave functions for the ground state and the two lowest
excited states: $\lambda_0$, $\lambda_1$ and $\lambda_2$

The Jacobi Algorithm we used correspond to doing
$$
\mathbf{B} = \mathbf{S}^T_N \cdots \mathbf{S}^T_2 \mathbf{S}^T_1 \mathbf{S}^T_0
             \mathbf{A}
             \mathbf{S}_0 \mathbf{S}_1 \mathbf{S}_2 \cdots \mathbf{S}_N
           = \mathbf{S}^T A \mathbf{S}
$$
where $N$ is the number of iterations that was required to make all the
non-diagonal elements of $\mathbf{B}$ be $\approx 0$ and $\mathbf{S}$ is the
eigenvectors, or eigenfunctions, that correspond to the eigenvectors in
$\mathbf{B}$.


\section{Results}
Now that our eigenvalue solver is implemented and working, we can look at how
the results vary for the different parameters $n_{step}$ and  $\rho_{max}$.

First, we decide on a fixed value for $n_{step}$, say $ n_{step} = 100$. By
running the algorithm for various $\rho_{max}$, we can find out approximately
for what $\rho_{max}$ we get the best results. 


% paulms:

\bildeto{images/prob_1}{images/prob_2}{fig:prob12}{Probability
distribution for the three lowest energy states for two electrons
in two different weak potentials.}
\bildeto{images/prob_3}{images/prob_4}{fig:prob34}{Two different
stronger potentials.}

From figures \ref{fig:prob12} and \ref{fig:prob34} we see how the
electrons behave in oscillators of different potential strengths.
Note that the radius $\rho$ is different on each figure. For each
different $\omega_r$ $\rho_{max}$ was selected so that the wave
function settled at $\approx 0$ \emph{before} $\rho_{max}$. The
ground state ($E_0$) has one top, the first excited state ($E_1$)
has two tops, $E_2$ has three top, etc.

For higher $\omega_r$ the ``tops'' are not necessarily extremum
points, but rather ``humps'' in the slopes, and it it is only when
$\omega_r$ that we can see a decrease in probablity near $\rho = 0$
where the electrons are close together. One could expect a more
distinct valley here, but it seems the stronger potentials dominate
the Coulomb force.

The strange behavior at the far left side is probably just due to
the sloppy method used to avoid division by zero.

\bildeto{images/prob_0_short}{images/prob_0_long}{fig:prob0}{Two
    electrons in oscillators of different sizes with \emph{no}
potential strength at all.}

In figure \ref{fig:prob0} there are only two ``free'' electrons.
Obviously from the figures, they still appear to be in a potential,
but no matter how large we set $\rho_{max}$ they will use all the
space available, since the only force present is the Coulomb force
that is pushing them away from eachother. This strange behavior
occurs because our model will always set the probability to 0 at
$\rho_{max}$ and not consider the possibility that there is some
probability of existance beyond this.
 % paulms' results

For one electron the Jacobi eigenvalue algorithm calculated the
three lowest eigenvalues as shown in the following table.
\begin{center}
    \begin{tabular}{|c|c|c|c|}
        \hline
        $n$ & $\lambda_1$ & $\lambda_2$ & $\lambda_3$ \\
        \hline
        \hline
        $50$ & 2.99757 & 6.98794 & 10.9759 \\
        \hline
        $100$ & 2.99938 & 6.99702 & 10.9981 \\
        \hline
        $200$ & 2.99984 & 6.99934 & 11.0038 \\
        \hline
        $300$ & 2.99993 & 6.99978 & 11.0049 \\
        \hline
    \end{tabular}
\end{center}
And the following values for two electrons with the Coulomb
interaction.
\begin{center}
    \begin{tabular}{|c|p{1.5cm}|p{1.5cm}|p{1.5cm}|p{1.5cm}|}
        \hline
        $n$ & $\omega_r = 0.01$ & $\omega_r = 0.5$ & $\omega_r = 1$
        & $\omega_r = 5$ \\
        \hline
        \hline
        $50$ & 0.986125, 2.61027, 5.13557 & 2.23541, 4.26627, 6.85092 &
        4.05533, 7.89821, 11.7991 & 17.3847, 36.7693, 56.1131 \\
        \hline
        $100$ & 0.986314, 2.61246, 5.14571 & 2.23587, 4.26892, 6.86137
        & 4.05723, 7.90693, 11.8206 & 17.4324, 36.9952, 56.667 \\
        \hline
        $200$ & 0.986362, 2.61302, 5.1483 & 2.23599, 4.26959, 6.86404 &
        4.05771, 7.90916, 11.8261 & 17.4446, 37.0525, 56.8068 \\
        \hline
        $300$ & 0.986372, 2.61312, 5.14879 & 2.23602, 4.26972, 6.86454
        & 4.05781, 7.90957, 11.8271 & 17.4469, 37.0632, 56.8329 \\
        \hline
    \end{tabular}
\end{center}


\section{B (placeholder, remove it if you want to, see if I care)}
Now that our eigenvalue solver is implemented and working, we can look at how
the results vary for the different parameters $n_{step}$ and  $\rho_{max}$.

First, we decide on a fixed value for $n_{step}$, say $ n_{step} = 100$. By
running the algorithm for various $\rho_{max}$, we can find out approximately
for what $\rho_{max}$ we get the best results. 



\section{Discussion}
% Remove line below to get rid of placeholder text
\lipsum[1]

\section{Code}
All code can be found in our GitHub repository. The individual
repositories are added as submodules.

\url{https://github.com/PaulMag/FYS3150_project2}


\end{document}
